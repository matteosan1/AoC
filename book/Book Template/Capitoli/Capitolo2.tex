
\chapter{Nuovi comandi introdotti in 'Custom.sty'}
\Minipage{.45}{\begin{longtable}{r|r|r}
Mathbb ($\setminus$A)& Mathcal ($\setminus$AA) & Mathscr ($\setminus$AAA)\\\hline
$\A$ & $\AA$ & $\AAA$\\[0.1cm]
$\B$ & $\BB$ & $\BBB$\\[0.1cm]
$\C$ & $\CC$ & $\CCC$\\[0.1cm]
$\D$ & $\DD$ & $\DDD$\\[0.1cm]
$\E$ & $\EE$ & $\EEE$\\[0.1cm]
$\F$ & $\FF$ & $\FFF$\\[0.1cm]
$\G$ & $\GG$ & $\GGG$\\[0.1cm]
$\H$ & $\HH$ & $\HHH$\\[0.1cm]
$\I$ & $\II$ & $\III$\\[0.1cm]
$\J$ & $\JJ$ & $\JJJ$\\[0.1cm]
$\K$ & $\KK$ & $\KKK$\\[0.1cm]
$\L$ & $\LL$ & $\LLL$\\[0.1cm]
$\M$ & $\MM$ & $\MMM$\\[0.1cm]
$\N$ & $\NN$ & $\NNN$\\[0.1cm]
$\O$ & $\OO$ & $\OOO$\\[0.1cm]
$\P$ & $\PP$ & $\PPP$\\[0.1cm]
$\Q$ & $\QQ$ & $\QQQ$\\[0.1cm]
$\R$ & $\RR$ & $\RRR$\\[0.1cm]
$\S$ & $\SS$ & $\SSS$\\[0.1cm]
$\T$ & $\TT$ & $\TTT$\\[0.1cm]
$\U$ & $\UU$ & $\UUU$\\[0.1cm]
$\V$ & $\VV$ & $\VVV$\\[0.1cm]
$\W$ & $\WW$ & $\WWW$\\[0.1cm]
$\X$ & $\XX$ & $\XXX$\\[0.1cm]
$\Y$ & $\YY$ & $\YYY$\\[0.1cm]
$\Z$ & $\ZZ$ & $\ZZZ$\\[0.1cm]
\end{longtable}
}{.35}{
\begin{longtable}{rl}
& \bf Simboli\\\hline
\text{$\setminus$d}& $\d$ \\[0.1cm]
\text{$\setminus$e}& $\e$ \\[0.1cm]
\text{$\setminus$exp}& $\exp$ \\[0.1cm]
\text{$\setminus$f}& $\f$ \\[0.1cm]
\text{$\setminus$g}& $\g$ \\[0.1cm]
\text{$\setminus$i}& $\i$ \\[0.1cm]
\text{$\setminus$Id\{A\}}& $\Id{A}$ \\[0.1cm]
\text{$\setminus$j}& $\j$ \\[0.1cm]
\text{$\setminus$l}& $\l$ \\[0.1cm]
\text{$\setminus$sbullet}& $\sbullet$ \\[0.1cm]
\text{$\setminus$v}& $\v$ \\[0.1cm]
\text{$\setminus$Zero}& $\Zero$ \\[0.1cm]
\end{longtable}
\begin{longtable}{rl}
& \bf Operatori\\\hline
\text{$\setminus$*}& $\*$ \\[0.1cm]
\text{$\setminus$CF\{A\}\{B\}}& $\CF{A}{B}$ \\[0.1cm]
\text{$\setminus$codim\{A\}\{B\}}& $\codim{A}{B}$ \\[0.1cm]
\text{$\setminus$Dif\{A\}\{B\}}& $\Dif{A}{B}$ \\[0.1cm]
\text{$\setminus$Diff\{A\}\{B\}}& $\Diff{A}{B}$ \\[0.1cm]
\text{$\setminus$Floor\{A\}}& $\Floor{A}$ \\[0.1cm]
\text{$\setminus$Imm\{A\}}& $\Imm{A}$ \\[0.1cm]
\text{$\setminus$Leg\{A\}\{B\}}& $\Leg{A}{B}$ \\[0.1cm]
\text{$\setminus$nobarfrac\{A\}\{B\}}& $\nobarfrac{A}{B}$ \\[0.1cm]
\text{$\setminus$Norma\{A\}}& $\Norma{A}$ \\[0.1cm]
\text{$\setminus$Lie\{A\}\{B\}}& $\Lie{A}{B}$ \\[0.1cm]
\text{$\setminus$PS\{A\}\{B\}}& $\PS{A}{B}$ \\[0.1cm]
\text{$\setminus$Rad\{A\}}& $\Rad{A}$ \\[0.1cm]
\text{$\setminus$Re\{A\}}& $\Re{A}$ \\[0.1cm]
\text{$\setminus$Res\{A\}\{B\}}& $\Res{A}{B}$ \\[0.1cm]
\text{$\setminus$Stirling\{A\}\{B\}}& $\Stirling{A}{B}$ \\[0.1cm]
\text{$\setminus$Val\{A\}\{B\}}& $\Val{A}{B}$ \\[0.1cm]
\end{longtable}
}
\newpage
\begin{longtable}{rl}
&\bf Insiemi e gruppi\\\hline
\text{$\setminus$Ann\{A\}}& $\Ann{A}$ \\[0.1cm]
\text{$\setminus$Aut\{A\}}& $\Aut{A}$ \\[0.1cm]
\text{$\setminus$Fix\{A\}}&$\Fix{A}$ \\[0.1cm]
\text{$\setminus$GAut\{A\}}& $\GAut{A}$ \\[0.1cm]
\text{$\setminus$GL\{A\}\{B\}}& $\GL{A}{B}$ \\[0.1cm]
\text{$\setminus$GLE\{A\}}& $\GLE{A}$ \\[0.1cm]
\text{$\setminus$Hom\{A\}\{B\}\{C\}}& $\Hom{A}{B}{C}$ \\[0.1cm]
\text{$\setminus$Im\{A\}}& $\Im{A}$ \\[0.1cm]
\text{$\setminus$Ker\{A\}}& $\Ker{A}$ \\[0.1cm]
\text{$\setminus$OM\{A\}\{B\}}& $\OM{A}{B}$ \\[0.1cm]
\text{$\setminus$Seq\{A\}\{B\}\{C\}}& $\Seq{A}{B}{C}$ \\[0.1cm]
\text{$\setminus$Set\{A\}}& $\Set{A}$ \\[0.1cm]
\text{$\setminus$SLM\{A\}\{B\}}& $\SLM{A}{B}$ \\[0.1cm]
\text{$\setminus$SO\{A\}\{B\}}& $\SO{A}{B}$ \\[0.1cm]
\text{$\setminus$Span\{A\}}& $\Span{A}$ \\[0.1cm]
\text{$\setminus$SU\{A\}\{B\}}& $\SU{A}{B}$ \\[0.1cm]
\text{$\setminus$Stab\{A\}\{B\}}& $\Stab{A}{B}$ \\[0.1cm]
\text{$\setminus$TM\{A\}\{B\}}& $\TM{A}{B}$ \\[0.1cm]
\text{$\setminus$Tr\{A\}}& $\Tr{A}$ \\[0.1cm]
\text{$\setminus$UM\{A\}\{B\}}& $\UM{A}{B}$ \\[0.1cm]
\end{longtable}
%%%%%%%%%%%%%%%%%%%%%%%%%%%%%%%%%%%
\section{Ambienti equazione}
\begin{verbatim}
\EqL{Equazione con label}{B}
\end{verbatim}
\EqL{Equazione\ con\ label}{B}

\begin{verbatim}\Eq{A & B\\ C & D}\end{verbatim}
\Eq{A& B\\C&D}
%%%%%%%%%%%%%%%%%%%%%%%%%%%%%%%%%%%
\section{Ambienti funzione}	
\begin{verbatim}
\Map{A}{B}{C}{D}{E}
\end{verbatim}
$$\Map{A}{B}{C}{D}{E}$$

\begin{verbatim}
\NMap{A}{B}{C}{D}
\end{verbatim}
$$\NMap{A}{B}{C}{D}$$
%%%%%%%%%%%%%%%%%%%%%%%%%%%%%%%%%%%
\section{Ambiente introduzione ai capitoli (ad esempio)}
\begin{verbatim}
\begin{TitoloIntro}[colbacktitle=red, width=\textwidth]{Titolo}{Riquadro}\end{TitoloIntro}
\end{verbatim}
\begin{TitoloIntro}[colbacktitle=red, width=\textwidth]{Titolo}{Riquadro}\end{TitoloIntro}

\begin{verbatim}
\begin{RTitoloIntro}[colbacktitle=red]{Titolo}{Riquadro}\end{RTitoloIntro}
\end{verbatim}
\begin{RTitoloIntro}[colbacktitle=red]{Titolo}{Riquadro}\end{RTitoloIntro}
%%%%%%%%%%%%%%%%%%%%%%%%%%%%%%%%%%%
\section{Ambiente minipage}
\begin{verbatim}
\Minipage{0.5}{Pagina a sinistra}{0.5}{Pagina a destra}
\end{verbatim}
\Minipage{0.5}{Pagina a sinistra}{0.5}{Pagina a destra}	
%%%%%%%%%%%%%%%%%%%%%%%%%%%%%%%%%%%
\section{Citazione}
\begin{verbatim}
\Quote{A}{B}{.5}
\end{verbatim}
\Quote{A}{B}{.5}
%%%%%%%%%%%%%%%%%%%%%%%%%%%%%%%%%%%
\section{Pacchetto "polynomial"}
\paragraph*{}Offre un migliore typesetting dei polinomi (si scrivono solo i coefficienti: la variabile e gli esponenti sono gestiti con le opzioni all'inizio del comando e non si devono scrivere in ogni monomio). Una lista esaustiva di comandi disponibili è la seguente:
\begin{center}
\begin{minipage}{.7\textwidth}
\begin{verbatim}
\polynomial{1,1,2,c,0,0,0,0,5}
\end{verbatim}
%Lista dei coefficienti di x^0, x^1,x^2 ecc
\end{minipage}
\begin{minipage}{.2\textwidth}
\Eq{\polynomial{1,1,2,c,0,0,0,0,5}}
\end{minipage}
\begin{minipage}{.7\textwidth}
\begin{verbatim}
\polynomialfrac{1,1,2,c,0,0,0,0,5}{0,1,1,4}
\end{verbatim}
%Funzione razionale: elenca i coefficienti
\end{minipage}
\begin{minipage}{.2\textwidth}
\Eq{\polynomialfrac{1,1,2,c,0,0,0,0,5}{0,1,1,4}}
\end{minipage}
\begin{minipage}{.7\textwidth}
\begin{verbatim}	
\polynomial[falling]{a,b,c,-d,e}
\end{verbatim}
\end{minipage}
\begin{minipage}{.2\textwidth}\vspace{-.1cm}
\Eq{\hspace{-1cm}\polynomial[falling]{a,b,c,-d,e}}
\end{minipage}
\begin{minipage}{.6\textwidth}
\begin{verbatim}
\polynomial[reciprocal]{a,b,c,d,e}
\end{verbatim}
\end{minipage}
\begin{minipage}{.3\textwidth}
\Eq{\polynomial[reciprocal]{a,b,c,d,e}}
\end{minipage}
\begin{minipage}{.7\textwidth}
\begin{verbatim}
\polynomial[var=t]{1,1,2,c,0,0,0,0,5}
\end{verbatim}
\end{minipage}
\begin{minipage}{.2\textwidth}
\Eq{\polynomial[var=t]{1,1,2,c,0,0,0,0,5}}
\end{minipage}
\begin{minipage}{.6\textwidth}
\begin{verbatim}
\polynomial[start=1000]{1,1,2,c,0,0,0,0,5}
\end{verbatim}
\end{minipage}
\begin{minipage}{.35\textwidth}
\Eq{\polynomial[start=1000]{1,1,2,c,0,0,0,0,5}}
\end{minipage}
\begin{minipage}{.7\textwidth}
\begin{verbatim}
\polynomial[step=-5]{1,1,2,c}
\end{verbatim}
\end{minipage}
\begin{minipage}{.2\textwidth}
\Eq{\polynomial[step=-5]{1,1,2,c}}
\end{minipage}
\begin{minipage}{.7\textwidth}
\begin{verbatim}\polynomial[add=\otimes]{1,1,2,c,0,0,0,0,5}
\end{verbatim}
\end{minipage}
\begin{minipage}{.2\textwidth}
\Eq{\polynomial[add=\otimes]{1,1,2,c,0,0,0,0,5}}
\end{minipage}
\begin{minipage}{.7\textwidth}
\begin{verbatim}\polynomial[sub=\times]{1,2,-c,0,0,0,0,-5}
\end{verbatim}
\end{minipage}
\begin{minipage}{.2\textwidth}
\Eq{\polynomial[sub=\times]{1,2,-c,0,0,0,-5}}
\end{minipage}
\end{center}
%%%%%%%%%%%%%%%%%%%%%%%%%%%%%%%%%%%
\section{Pacchetto "accents"}
\paragraph*{} Migliora il typesetting di lettere accentate in ambiente matematico. Ad esempio
\begin{center}
\begin{minipage}{.4\textwidth}
\begin{verbatim}
\undertilde{Prova}
\end{verbatim}
\end{minipage}
\begin{minipage}{.2\textwidth}
\Eq{\undertilde{Prova}}
\end{minipage}
\begin{minipage}{.4\textwidth}
\begin{verbatim}
\underaccent{under}{a}
\end{verbatim}
\end{minipage}
\begin{minipage}{.2\textwidth}
\Eq{\underaccent{under}{a}}
\end{minipage}
\begin{minipage}{.4\textwidth}
\begin{verbatim}
\accentset{over}{a}
\end{verbatim}
\end{minipage}
\begin{minipage}{.2\textwidth}
\Eq{\accentset{over}{a}}
\end{minipage}
\begin{minipage}{.4\textwidth}
\begin{verbatim}
\hat{\accentset{\pi}{a}}
\end{verbatim}
\end{minipage}
\begin{minipage}{.2\textwidth}
\Eq{\hat{\accentset{over}{a}}}
\end{minipage}
\begin{minipage}{.4\textwidth}
\begin{verbatim}
\ring{a}
\end{verbatim}
\end{minipage}
\begin{minipage}{.2\textwidth}
\Eq{\ring{a}}
\end{minipage}
\begin{minipage}{.4\textwidth}
\begin{verbatim}
\overrightarrow{Testo}
\end{verbatim}
\end{minipage}
\begin{minipage}{.2\textwidth}
\Eq{\overrightarrow{Testo}}
\end{minipage}
\begin{minipage}{.4\textwidth}
\begin{verbatim}
 \underrightarrow{Testo}
\end{verbatim}
\end{minipage}
\begin{minipage}{.2\textwidth}
\Eq{ \underrightarrow{Testo}}
\end{minipage}
\begin{minipage}{.4\textwidth}
\begin{verbatim}
\xrightarrow[Sotto]{Sopra}
\end{verbatim}
% da usare al posto di overset quando c'è un commento sopra/sotto una freccia in una equazione
\end{minipage}
\begin{minipage}{.2\textwidth}
\Eq{\xrightarrow[Sotto]{Sopra}}
\end{minipage}
\begin{minipage}{.4\textwidth}
\begin{verbatim}
\overset{Sopra}{Testo}
\end{verbatim}
\end{minipage}
\begin{minipage}{.2\textwidth}
\Eq{\overset{Sopra}{Testo}}
\end{minipage}
\begin{minipage}{.4\textwidth}
\begin{verbatim}
\underset{Sotto}{Testo}
\end{verbatim}
\end{minipage}
\begin{minipage}{.2\textwidth}
\Eq{\underset{Sotto}{Testo}}
\end{minipage}
\begin{minipage}{.4\textwidth}
\begin{verbatim}
$$ \sideset{_a^b}{_c^d}\sum$$
\end{verbatim}
% Per mettere accenti vicino a simboli di sum, prod, bigoplus, bigotimes, ecc.
\end{minipage}
\begin{minipage}{.2\textwidth}
\Eq{\sideset{_a^b}{_c^d}\sum}\vspace{1cm}
\end{minipage}
\end{center}
%%%%%%%%%%%%%%%%%%%%%%%%%%%%%%%%%%%
\section{Typesetting di formule matematiche in \$ \$}
\begin{verbatim}
$\tfrac{Numeratore}{Denominatore}$
\end{verbatim}
Testo $\tfrac{Numeratore}{Denominatore}$ Testo

\begin{verbatim}
$\tbinom{N}{D}$
\end{verbatim}
Testo $\tbinom{N}{D}$ Testo